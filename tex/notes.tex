\documentclass{article}
\usepackage{minted}

\begin{document}

\section{Who Am I?}

I'm Matthew Balch. My career has included numerous computer languages such as
C, C++, Javascript, Perl, Python, PHP, and many others. I've been programming 
in Julia for around a year for work projects as an alternative to R, Python,
and MATLAB. I also develop the GtkBuilderAid package for easing use of GTK
Glade interface files within Julia.

\section{What is Julia?}

Julia was not meant as a general purpose programming language, but you wouldn't
know that if someone didn't tell you. The use case Julia is designed around is 
high performance technical computing as seen for physics simulations,"big data"
analysis, and bioinformatics. The languages currently filling those niches were
either too slow at runtime with Python, MATLAB, and R, or too slow at 
development time with C. The goal, in short, was to deliver a language with the
performance of a low-level language and the ease-of-use of a high level
language.

\section{Some Caution About This Presentation}

This presentation was written for people who already know how to program. Julia
may or may not be a good beginner's language, but this is not a beginner's
presentation.

While creating this presentation I realized that fully describing a computer
language is a lengthy task that I can't achieve within even the most patient
person's attention span. For this reason I'll mostly gloss over points where
Julia behaves identically to most other languages, such as infix arithmetic,
function call styling, and control flow. Points that differ slightly, such
as type annotation, for loops, and collections, will be mentioned in passing.
Points that differ significantly, such as metaprogramming, iterators, will 
have more focus.

\section{A Quick Test to Get Started}

Before moving on let's all implement FizzBuzz.

Making a one-liner FizzBuzz in Julia without repetition is possible.

A hint:
Use the "any" function and boolean short-circuiting to create a one-liner
without any repetition.

\begin{minted}{julia}
[begin 
  (any((
    i % 3 == 0 && print("Fizz") == nothing, 
    i % 5 == 0 && print("Buzz") == nothing)) || 
  print(i) == nothing); println()
end for i in 1:100]
\end{minted}

That's the basics of what you need to know to just use Julia, so we'll move
on to proper development next.

\section{Let's Look at Some Code}

Look at an existing package 

\section{Something to Work On}

Just talking about code isn't sufficient. In order to learn, practice is 
necessary. I'm here to help everyone get started on whatever project they
want within Julia. Starting by contributing to an existing package is
advisable.

\end{document}

