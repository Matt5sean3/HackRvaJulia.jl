\documentclass{beamer}

\usepackage[utf8]{inputenc}
\usepackage{minted}

\title[Julia]{Introduction to the Julia Programming Language}
\author{Matthew Balch}
\institute{HackRVA}
\date{September 17, 2016}

\begin{document}

\frame{\titlepage}

\begin{frame}
\frametitle{Who Am I?}
I'm Matthew Balch. My career has included numerous computer languages with 
Perl, Python, PHP, C, C++, Javascript, and many others. I've been programming 
in Julia for around a year for work projects as an alternative to R, Python, and MATLAB.
I also maintain the GtkBuilderAid package for easing use of GTK Glade interface 
files within Julia.

\end{frame}

\begin{frame}
\frametitle{You all installed Julia, right?}
\begin{itemize}
  \item If you're on Linux, use your package manager
  \item If you're on Windows, go to the downloads page at julialang.org
  \item If you're on Mac, you're on your own
\end{itemize}
\end{frame}

\begin{frame}
\frametitle{Start Julia}

\begin{itemize}
  \item In Linux, from the command line type "julia"
  \item In Windows, use the start menu to open the REPL
  \item In Mac, open programs however as you usually do in Mac
\end{itemize}

\end{frame}

\begin{frame}[fragile]
\frametitle{Lets Get Started}

\begin{minted}{julia}
Pkg.clone("https://github.com/Matt5sean3/HackRvaJulia.jl.git")
using HackRvaJulia
start_intro()
\end{minted}

\begin{enumerate}
  \item Installs the HackRvaJulia package
  \item Loads the HackRvaJulia package into the environment
  \item Calls the start\_intro function
\end{enumerate}

\end{frame}

\end{document}

