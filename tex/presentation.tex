\documentclass{beamer}

\usepackage[utf8]{inputenc}
\usepackage{minted}
\usepackage{xcolor}
\definecolor{bgcolor}{RGB}{5, 218, 186}
\setbeamercolor{background canvas}{bg=bgcolor}

\title[Julia]{Introduction to the Julia Programming Language}
\author{Matthew Balch}
\institute{HackRVA}
\date{September 17, 2016}

\begin{document}

\frame{\titlepage}

\begin{frame}
\frametitle{Why Program in Julia?}
\begin{itemize}
  \item Speed
  \item Familiar syntax
  \item A growing library of packages
  \item Excellent integration with C and Fortran binaries
  \item Metaprogramming
  \item Parallel programming
\end{itemize}
\end{frame}

\begin{frame}
\frametitle{You all installed Julia, right?}
\begin{itemize}
  \item If you're on Linux, use your package manager
  \item If you're on Windows, go to the downloads page at julialang.org
  \item If you're on Mac, you're on your own
  \item If you don't have your computer, you've got bigger problems.
\end{itemize}
\end{frame}

\begin{frame}
\frametitle{Start Julia}
\begin{itemize}
  \item In Linux, from the command line type "julia".
  \item In Windows, use the start menu to open the REPL.
  \item In Mac, open programs however you usually do in Mac.
  \item A desktop launcher should work in whatever operating system you use.
\end{itemize}
\end{frame}

\begin{frame}
\frametitle{Quick Ways to Learn More}
\begin{itemize}
  \item Type "?" then the function name to see documentation
  \item Go to docs.julialang.org
  \item Say something if I'm going to fast
\end{itemize}
\end{frame}

\begin{frame}[fragile]
\frametitle{Lets Get Started}
\begin{minted}{julia}
Pkg.clone("https://github.com/Matt5sean3/HackRvaJulia.jl.git")
using HackRvaJulia
start_intro()
\end{minted}
\begin{enumerate}
  \item Installs the HackRvaJulia package
  \item Loads the HackRvaJulia package into the environment
  \item Calls the start\_intro function
  \item Follow the instructions in the GUI
\end{enumerate}
\end{frame}

\begin{frame}[fragile]
\frametitle{Hello World!}
\begin{minted}{julia}
function hello(name::AbstractString = "World")
  println("Hello $name!")
end

hello(name::Symbol) =
  hello(string(name))

hello(name, times::Integer) =
  [hello(name) for i in 1:times]

hello()
hello("HackRVA")
hello("HackRVA", 5)
\end{minted}
\end{frame}

\begin{frame}[fragile]
\frametitle{Hello World! (cont.)}
These lines alone show us quite a lot.

Type Annotation
\begin{minted}{julia}
name::AbstractString
name::Symbol
\end{minted}

Dollar-sign substitution in strings

Semi-colons not needed, newlines mean something
\begin{minted}{julia}
"Hello $name!"
\end{minted}

long form function definition
\begin{minted}{julia}
function hello(name::AbstractString = "World")
  println("Hello $name!")
end
\end{minted}

short form function definition
\begin{minted}{julia}
hello(name::Symbol) =
  hello(string(name))
\end{minted}

begin-end style code blocks
\end{frame}

\begin{frame}[fragile]
\frametitle{Return Values}
Every function and code block will has a return value. If no return statement
is used then the last statement is the return value. This allows for 
surprisingly compact code quite often.

\begin{minted}{julia}
point1 = (3, 3)
point2 = (6, 7)
function distance(p1, p2)
  (p1[1] - p2[1])^2
end
distance(point1, point2) == 5
\end{minted}

\end{frame}

\begin{frame}[fragile]
\frametitle{Type System}

Types are either concrete or abstract. A sort of type tree can be generated
using abstract types. A concrete type cannot be have a sub-type.

\begin{minted}{julia}
abstract AnAbstractType <: ASuperType

type AConcreteType <: AnAbstractType
  unannotated_member
  annotated_member::SomeType
end

\end{minted}

\end{frame}

\begin{frame}
\frametitle{Multiple Dispatch}

Rather than objects having methods, functions have methods.

When a function is called the types of the input arguments are used to decide 
which method of the function is used.

\end{frame}

\begin{frame}[fragile]
\frametitle{Method Ambiguity}

When there is ambiguity as to which method would be used, the julia interpreter
will tell you at definition time. When this happens, adding another method to 
resolve the ambiguity should fix everything.

\begin{minted}{julia}
# This causes an ambiguity warning
f(x::Float64, y) = println("First argument is a float")
f(x, y::Float64) = println("Second argument is a float")
\end{minted}

\begin{minted}{julia}
# This won't cause an ambiguity warning
f(x::Float64, y::Float64) = println("Both arguments are floats")
f(x::Float64, y) = println("Only first argument is a float")
f(x, y::Float64) = println("Only second argument is a float")
\end{minted}
\end{frame}

\begin{frame}[fragile]
\frametitle{Control Flow}
Essentially the same as most other languages

If statements only handle the boolean type. Any other type will throw an error.
\begin{minted}{julia}
if condition
  # happens if condition is true
end

if 1 # Will throw a type error
  # something-something
end

if value == 0
  # do something for the value being 0
elseif value < 0
  # do something for the value being below 0
else
  # do something for the value being over 0
end

# if statements can return values too
x = if value == 0
  0
elseif value < 0
  -1
else
  1
end

# ternary operator is also supported
x = (condition)? iftrue : iffalse 
\end{minted}

\end{frame}

\begin{frame}[fragile]
\frametitle{Control Flow (cont.)}

\begin{minted}{julia}
# if statements can return values too
x = if value == 0
  0
elseif value < 0
  -1
else
  1
end

# ternary operator is also supported
x = (condition)? iftrue : iffalse 
\end{minted}
\end{frame}

\begin{frame}[fragile]
\frametitle{Control Flow (cont.)}

For and while loops are also an option

\begin{minted}{julia}
# while loops are available
while keep
  # hopefully the condition changes
end

# while blocks always return "nothing"
x = while condition
  # hopefully this loop exits somehow
end

# for blocks also return nothing
x = for i in 1:5
  # happens 5 times
  # return value of the block doesn't go anywhere
  not_returned_value
end
\end{minted}
\end{frame}

\begin{frame}[fragile]
\frametitle{Control Flow (cont.)}

\begin{minted}{julia}
# However, comprehensions return an array
x = [2 for i in 1:5]

# Comprehensions allows simulating for blocks that return lists
x = [begin
  # do something to compute the element
  list_element
end for i in iterable]
\end{minted}

\end{frame}

\begin{frame}
\frametitle{Let's Make Sure We Know the Basics}
You should all now know enough to implement FizzBuzz

\begin{itemize}
  \item Print the numbers 1 through 100
  \item for multiples of 3 print Fizz instead of the number
  \item for multiples of 5 print Buzz instead of the number
  \item for multiples of both 3 and 5 print FizzBuzz instead of the number
  \item for extra credit implement it as one line
  \item to prove you're a badass do it without repetition
\end{itemize}

\end{frame}

\begin{frame}[fragile]
\frametitle{Iterables}

Iterables are very useful

To use something as an iterable the start, next, and done methods must be 
defined

\begin{minted}{julia}
# returns the initial state for iteration
start(iter) = initial_state

# returns a tuple containing the item in the iterator and the next 
# iteration state
next(iter, state) = (item, next_state)

# returns whether iteration has completed
done(iter, state) = is_finished

# A length method is necessary to use the iterable in a comprehension
length(iter) = number_of_times_iterating
\end{minted}

\end{frame}

\begin{frame}[fragile]
\frametitle{Packages}

Julia's package system is built around git. If you don't know git this could 
get confusing.

There are a few basic package commands for retrieving packages.
\begin{minted}{julia}
Pkg.init() # Initial configuration of the package manager
Pkg.add("package_name") # Install the named package
Pkg.rm("package_name") # Uninstall the named package
Pkg.generate("package_name", "license") # Create a new package with that name
\end{minted}

\end{frame}

\begin{frame}[fragile]
\frametitle{Using Packages}

Like most languages, to do much beyond the basics using just what's built-in
would be ridiculous.

\begin{minted}{julia}
use JSON

import Base: start, next, done, length

\end{minted}

One important thing to know is that more methods can be added to imported
functions but not to functions brought into scope via "use". The Core and
Base packages are always implicitly used.

\end{frame}

\begin{frame}
\frametitle{Let's Make a Package}

\begin{enumerate}
  \item If you don't have a Github account, make one now
  \item Think about what the package should do
  \item Create a new Julia package named accordingly
\end{enumerate}

If you have questions about using Julia, git, or anything else now is the time 
to ask.
\end{frame}

\begin{frame}[fragile]
\frametitle{Generate a Package}

\begin{minted}{julia}
Pkg.generate("PackageName", "License")
\end{minted}

\begin{itemize}
  \item A few initial files will be created and a git repository will be initialized.
  \item A few license strings are recognized: MIT, BSD, and ASL
  \item The package's location is found with Pkg.dir("PackageName")
\end{itemize}

\end{frame}

\begin{frame}
\frametitle{Basics of a Package}

\begin{itemize}
  \item The REQUIRE file lists the dependencies of the package
  \item README.md and LICENSE.md are automatically added
  \item A test directory and src directory with basic templates are created
  \item Continuous Integration files for Travis-CI (.travis.yml) and Appveyor (appveyor.yml) are made
\end{itemize}

\end{frame}

\begin{frame}
\frametitle{Making a Github Account}

\begin{enumerate}
  \item Go to github.com
  \item Click on Sign Up
  \item Go through the account creation process
  \item Add an SSL certificate
\end{enumerate}

\end{frame}

\begin{frame}
\frametitle{Configuring Git}

\begin{enumerate}
  \item Use the semi-colon to use a shell command in Julia
  \item git config --global user.name "Your Name"
  \item git config --global user.email "youremail@example.com"
  \item Connect using SSL keys
\end{enumerate}

\end{frame}

\begin{frame}
\frametitle{Ongoing Problems in Julia Land}

\begin{itemize}
  \item Backwards compatibility breaking changes are still happening
  \item There are still mistakes in even some fundamental packages
\end{itemize}

Julia is an open source project and the package system included in the
installation is well designed to make contributing a solution to these 
problems frictionless.

\end{frame}

\begin{frame}
\frametitle{Documentation Generation}

\begin{enumerate}
  \item documentation is important (if you didn't already know)
  \item In-code documentation keeps things simple
  \item Julia, like most languages, has a documentation generation package
  \item Install Documenter.jl
\end{enumerate}

\end{frame}

\end{document}

